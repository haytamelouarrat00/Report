\documentclass[a4paper,12pt]{report}
\usepackage[margin=2.5cm]{geometry}
\usepackage{graphicx}
\usepackage{fancyhdr}
\usepackage{tikz}
\usepackage{lipsum} % For filler text if needed
\usepackage{setspace}
\usepackage{titlesec}
\usepackage{everypage}

% --- No page numbering for title ---
\pagestyle{fancy}
\fancyhf{}
\fancyfoot[C]{\thepage}

% --- Bottom-right logo ---
\fancyfoot[R]{
    \raisebox{0.1\textwidth}{\includegraphics[height=0.25cm]{ressources/for_glory.png}}
}
\begin{document}

% --- Logo Placement with TikZ ---
\begin{tikzpicture}[remember picture, overlay]
    % Top-left: Company Logo
    \node[anchor=north west, xshift=1cm, yshift=-3cm] at (current page.north west) {
        \includegraphics[width=0.5\textwidth, keepaspectratio]{ressources/ss_horizontal_white_on_black.jpg} % <-- Replace with actual file
    };
    
    % Top-right: University Logo
    \node[anchor=north east, xshift=-1cm, yshift=-2cm] at (current page.north east) {
        \includegraphics[width=0.5\textwidth, keepaspectratio]{ressources/logo_upssitech.png} % <-- Replace with actual file
    };
\end{tikzpicture}

% ------------------------------------------------------------
\vspace*{4cm}

\begin{center}
    \Huge \textbf{Neural-Assisted Feature Matching}\\[1.5cm]
    \LARGE Internship Report\\[1.5cm]
    \large \textbf{Author:} EL OUARRAT Haytam\\
    \textbf{Internship Period:} Mars – August 2025\\
    \textbf{Location:} SteelSeries, Lille, France\\[2cm]
    
    \textbf{Advisors:} Pierre Biret, Damien Granger, Raphaël Greff\\
    \textbf{University Supervisor:} Phillipe Joly \\[2cm]
    
    \large Engineering Degree in Robotics and Interactive Systems \\
    UPSSITECH \\
    University of Toulouse
\end{center}

\vfill

\begin{center}
    \today
\end{center}

% --- Page Break 
\newpage
\tableofcontents
\listoffigures
% ------------------------------------------------------------
\chapter{Acknowledgements}
% ------------------------------------------------------------
\chapter{Introduction}

\section{Host Organism}
\subsection{Nahimic}
\subsection{SteelSeries}
\subsection{GN Group}
\subsection{Mission}

\section{Context \& Motivation}
\subsection{Role of Feature Matching in Computer Vision}
\subsection{Challenges in Gaming Applications}
\subsection{Limitations of Traditional Feature Matching Techniques}

\section{Project Objectives}
\subsection{Reprodusing Feature Matching Techniques with Neural Networks}
\subsection{Improving computational efficiency}
\subsection{Ensure matching accuracy for gaming footage}

\section{Industrial Relevance}
\subsection{Integration with SteelSeries Moments Software}
\subsection{Real-time performance constraints}

% ------------------------------------------------------------
\chapter{Literature Review}
\section{Traditional Feature Matching}
\subsection{Overview of SIFT, ORB, FAST}
\subsection{Comparative Strengths, Weaknesses, and Computational Costs}

\section{Neural Feature Matching}
\subsection{Review of Recent Methods: LoFTR, ALIKE, LightGlue, XFeat}

\section{Knowledge Distillation}
\subsection{Distillation Types: Response-Based, Feature-Based, Relation-Based}
\subsection{Applications in Model Compression and Matching Tasks}

\section{Lightweight Architectures for Edge Deployment}
\subsection{MobileNet, ShuffleNet, XFeat-Style Networks}
\subsection{Trade-Offs Between Efficiency and Accuracy}

\section{Gaps and Opportunities}
\subsection{Where Traditional Methods Fall Short}
\subsection{Where Neural Methods Remain Overkill for Real-Time CPU Usage}
\subsection{Motivation for a Hybrid/Distilled Approach}

% ------------------------------------------------------------
\chapter{Methodology}
\section{Problem Formulation}
\subsection{Define Feature Matching as Correspondence Prediction}
\subsection{Objectives in Terms of Speed, Accuracy, and Robustness}

\section{Baseline Selection}
\subsection{Justification for Using ORB or FAST as Teacher Models}
\subsection{Benchmark Datasets (e.g., HPatches, Gaming Clips)}

\section{Neural Architecture Design}
\subsection{Choice of Lightweight CNN or Transformer Backbone}
\subsection{Feature Extraction vs. Matching Separation}

\section{Distillation Strategy}
\subsection{Design of Teacher-Student Framework}
\subsection{Distillation Losses (e.g., L2 on Descriptors, Cross-Entropy on Match Maps)}

\section{Evaluation Metrics}
\subsection{Matching Precision, Recall, Repeatability}
\subsection{Runtime (FPS), Memory Footprint, CPU Load}

% ------------------------------------------------------------
\chapter{Implementation}
\section{Dataset Preparation}
\subsection{Gaming Video Frame Extraction}
\subsection{Synthetic Transformation Generation for Ground Truth Correspondences}

\section{Training Pipeline}
\subsection{Data Augmentation Strategies}
\subsection{Loss Function Components and Training Schedule}

\section{Model Optimization}
\subsection{Quantization, Pruning, or ONNX Export (if applicable)}
\subsection{Inference Optimization for CPU}

\section{Integration with SteelSeries Pipeline}
\subsection{Data Flow Alignment with Moments Software (if available)}
\subsection{Latency Tracking and Bottleneck Identification}

% ------------------------------------------------------------
\chapter{Results \& Analysis}
\section{Matching Quality}
\subsection{Quantitative Comparison with ORB, SIFT, and XFeat}
\subsection{Visual Results on Gaming Footage}

\section{Computational Efficiency}
\subsection{FPS and Latency Benchmarks}
\subsection{Memory and CPU Usage Profiles}

\section{Ablation Studies}
\subsection{Effect of Different Distillation Losses}
\subsection{Model Depth vs. Performance Trade-Offs}

\section{Real-Time Viability}
\subsection{End-to-End Latency Breakdown}
\subsection{Suitability for Gaming Hardware}

% ------------------------------------------------------------
\chapter{Conclusion and Future Work}
\section{Summary of Contributions}
\section{Limitations}
\subsection{Domain Generalization}
\subsection{Extreme Low-Light or High-Motion Scenes}

\section{Future Work}
\subsection{Self-Distillation or Online Distillation Strategies}
\subsection{Hardware-Specific Optimizations (e.g., ARM CPU Tuning)}
\subsection{Real-Time Deployment on End-User Devices}

% ------------------------------------------------------------
\chapter{References}
% ------------------------------------------------------------
\chapter{Appendices}
\section{Additional Figures}
\section{Code Snippets}
\section{Hyperparameter Tables}
\section{Hardware Specifications}
% ------------------------------------------------------------
\end{document}
